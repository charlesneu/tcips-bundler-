\section{Emulador Mininet}
\label{sec:mininet}

Mininet \cite{Handigol:2012} é um emulador de rede para prototipação em \gls{sdn}. A razão pela sua utilização deve-se ao fato de apenas alguns dispositivos de rede estarem disponíveis para \gls{sdn}, uma vez que ainda não é uma tecnologia difundida a nível industrial.  Além disso, a implementação de rede com elevado número de dispositivos de rede é muito difícil e dispendioso. Por isso, para contornar estes problemas, a virtualização foi realizada com a finalidade de prototipar e emular este tipo de tecnologia de rede e um dos mais importantes é o Mininet \cite{Wendong:2012}.  Mininet tem a capacidade de emular diferentes tipos de elementos de rede, tais como: \textit{host}, \textit{switches} (camada de enlace), roteadores (camada de rede) e conexões. Ele funciona em um único núcleo de Linux\cite{Negus:2015} e utiliza virtualização com a finalidade de emular uma rede completa que utiliza apenas um único sistema. No entanto, o \textit{host}, roteadores e links criados são elementos do mundo real, embora eles sejam criados por meio de software \cite{website:mininet}.

Criar uma rede no Mininet é relativamente simples. Pode-se usar linha de comando ou um componente chamado \textit{miniedit.py}, que implementa uma interface gráfica para o Mininet, este porém, possui algumas limitações em relação à linha de comando. Pela linha de comando, ao chamar o Mininet são passados os parâmetros sobre as características da rede como: topologia, número de \textit{hosts}, \textit{switches}, taxa de perda de pacotes, largura de banda, tipo de controlador, entre outros. O \textit{switch} padrão é o OpenSwitch \cite{Pettit:2010}, um \textit{switch} virtual desenvolvido especialmente para trabalhar com o protocolo Openflow. Para estudo mais aprofundado, recomenda-se a leitura da sua documentação em \cite{website:mininet}.