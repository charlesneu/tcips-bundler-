% Resumo em língua vernácula é um elemento obrigatório.
% Deve ser utilizado o ambiente abstract e o comando \keywords deve ficar no começo:

\begin{abstract}
  \keywords{IDS, IPS, OpenFlow, SDN, Segurança, Ataques por Varredura, \textit{Lightweight}}
A segurança tem sido uma das principais preocupações na comunidade de redes devido ao abuso de recursos e intrusão de fluxos maliciosos. Sistemas de Detecção de Intrusão, ou \gls{ids} e Sistemas de Prevenção de Intrusão, ou \gls{ips} já foram muito utilizados em redes tradicionais para prover a sua segurança. Porém, com o crescimento e a evolução da Internet, muitas alternativas acabaram não sendo utilizadas devido à falta de uma estrutura de desenvolvimento global e testes em ambientes reais. O surgimento do paradigma de Redes Definidas por \textit{Software}, ou \gls{sdn}, e do protocolo OpenFlow, trouxe maior flexibilidade para a programação de novos protocolos e realização de testes em ambientes reais, além da possibilidade de implementação gradativa em ambientes de produção. No entanto, a simples migração de alternativas de \gls{ids}/\gls{ips} tradicionais para ambientes \gls{sdn} não é eficaz o suficiente para detectar e prevenir ataques. Diversos trabalhos têm abordado o desenvolvimento de sistemas de detecção e prevenção de intrusão para \gls{sdn}, baseados em técnicas de análise de fluxos recebidos, o que ocasiona um aumento na latência de chaveamento por parte dos comutadores. Em alguns casos, há a duplicação de pacotes para que sejam analisados pelo \gls{ids}, o que ocasiona sobrecarga na rede, tornando o \gls{ids}/\gls{ips} um gargalo. Com base no contexto atual e beneficiado pelas características flexíveis de \gls{sdn}, este trabalho apresenta um novo IPS para \gls{sdn}, contra ataques \textit{port scan}, utilizando dados das tabelas de encaminhamento de \textit{switches} OpenFlow. Isto resulta em uma arquitetura não-intrusiva e \textit{lightweight}, com baixo consumo de recursos de banda da rede e de processamento. Os resultados mostram que nosso método foi efetivo na detecção de ataques \textit{port scan}.
\end{abstract}